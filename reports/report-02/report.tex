% !TeX program = xelatex
% !TeX encoding = UTF-8
\documentclass[a4paper, 12pt, answers]{exam}

\usepackage[brazil]{babel}
\let\latinencoding\relax
\usepackage[T1]{fontenc}

\usepackage[no-math]{fontspec}
\usepackage{newpxtext, newpxmath}
\usepackage[a4paper, margin=2cm]{geometry}

\usepackage{siunitx}
\usepackage{fancyvrb}
\usepackage{graphicx}
\usepackage{caption}

\input{definition.tex}

\title{\titulo}
\author{\nomeAutorUm e \nomeAutorDois}
\date{\today}

\sisetup{
  binary-units = true
}

\renewcommand{\solutiontitle}{}

\fvset{
  gobble = 8,
  frame = single
}

\newcommand{\printtitle}{
  \begin{center}
    {\Large \scshape \titulo}\\[1em]
    {\nomeAutorUm, \raAutorUm}\\
    {\nomeAutorDois, \raAutorDois}\\[1em]
    Professor: Dr\@. \nomeProfessor, \centroProfessor\\
    {\itshape \campusFaculdade}
  \end{center}
}

\begin{document}
  \printtitle
  
  \begin{questions}
    \question
    Seguir o tutorial de RMI.
    
    \begin{parts}
      \part
      Servidor: executar o \verb|rmi.bat| e depois o \verb|server.bat|.
      
      \begin{solution}
        \begin{Verbatim}[label={\$ server.bat}]
        ComputeEngine bound
        \end{Verbatim}
      \end{solution}
      
      \part
      Cliente: executar o \verb|client.bat|. Discorra sobre os tempos obtidos
      na execução de forma remota e local.
      
      \begin{solution}
      
        O procedimento utilizando a forma remota de execução do código é 
        consideravelmente mais lento. Isso implica que o \emph{overhead}
        gerado pela técnica de RMI assim como uma eventual latência de
        comunicação afetam de maneira enorme o custo da execução do código
        na ordem de segundos de diferença em relação a execução local, 
        ou seja, esta técnica de migração de código é custosa e quando 
        utilizada implicará em um decaimento significativo na performance 
        da aplicação, principalmente se a aplicação/código for relativamente
        simples como o caso do exemplo utilizado. Somando o observado com o
        fato de que a implementação utilizando RMI não é trivial, fica 
        demonstrado a dificuldade de implementação de sistemas de migração
        de código.
        
        
        \begin{Verbatim}[label={\$ client.bat}]
        *** REMOTE EXECUTION ***
        Pi: 3.141592653589793238462643383279502884197169399
        Locate RMI registry: 130 miliseconds.
        Lookup remote object by name in the registry: 119 miliseconds.
        Remote execution of the task: 53 miliseconds.
        Total elapsed time: 302 miliseconds.
        
        *** LOCAL EXECUTION ***
        Pi: 3.141592653589793238462643383279502884197169399
        Total elapsed time: 1 miliseconds.
        \end{Verbatim}
      \end{solution}
    \end{parts}
    
    \question
    Executar a classe \verb|serialization.SerialObject| com os parâmetros adequados
    (nome de arquivo e operação).
    
    \begin{parts}
      \part
      Qual o papel de cada operação (\verb|write|, \verb|read| e \verb|diff|)?
      
      \begin{solution}
        O programa utiliza um \verb|Properties|, que é uma classe que representa
        um conjunto de propriedades, que implementa \verb|Serializable|, permitindo
        que possa-se transmitir com o fluxo. Neste caso, obtém-se as propriedades
        do ambiente de execução do Java. Assim, o \verb|System.getProperties()| é
        importante para que se conheça as propriedades específicas de um
        dado sistema operacional.
        
        A operação \verb|diff| compara um \verb|Properties| em memória com outro,
        mostrando suas diferenças. A operação \verb|write| escreve as propriedades
        em disco. A operação \verb|read| lê as propriedades de um arquivo indicado
        como argumento de execução, recuperando o estado de execução em disco.
        
        \begin{Verbatim}[label={\$ java serialization.SerialObject obj.prop write}]
        java.runtime.name: Java(TM) SE Runtime Environment
        sun.boot.library.path: C:\Program Files\Java\jre1.8.0_211\bin
        java.vm.version: 25.211-b12
        java.vm.vendor: Oracle Corporation
        java.vendor.url: http://java.oracle.com/
        ...
        \end{Verbatim}
        
        \begin{Verbatim}[label={\$ java serialization.SerialObject obj.prop read}]
        java.runtime.name: Java(TM) SE Runtime Environment
        sun.boot.library.path: C:\Program Files\Java\jre1.8.0_211\bin
        java.vm.version: 25.211-b12
        java.vm.vendor: Oracle Corporation
        java.vendor.url: http://java.oracle.com/
        ...
        \end{Verbatim}
        
        \begin{Verbatim}[label={\$ java serialization.SerialObject obj.prop diff}]
        sun.java.command: serialization.SerialObject obj.prop diff 
        / serialization.SerialObject obj.prop write
        \end{Verbatim}
      \end{solution}
    \end{parts}
    
    \question
    Implementar um sistema cliente-servidor em que o cliente envia seu objeto
    Properties a um servidor por meio de \emph{socket}. O servidor compara
    o objeto enviado com seu próprio \emph{Properties} e devolve o resultado
    da operação \verb|diff| ao cliente, que é impresso na tela.
    
    \begin{solution}
      Arquivos de código-fonte anexados junto ao relatório no \verb|zip|.
    \end{solution}
  \end{questions}
\end{document}